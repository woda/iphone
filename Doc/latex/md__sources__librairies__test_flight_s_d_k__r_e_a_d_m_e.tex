The \hyperlink{interface_test_flight}{Test\-Flight} S\-D\-K allows you to track how beta testers are testing your application. Out of the box we track simple usage information, such as which tester is using your application, their device model/\-O\-S, how long they used the application, and automatic recording of any crashes they encounter.

The S\-D\-K can track more information if you pass it to \hyperlink{interface_test_flight}{Test\-Flight}. The Checkpoint A\-P\-I is used to help you track exactly how your testers are using your application. Curious about which users passed level 5 in your game, or posted their high score to Twitter, or found that obscure feature? See \char`\"{}\-Checkpoint A\-P\-I\char`\"{} down below to see how.

The S\-D\-K also offers a remote logging solution. Find out more about our logging system in the \char`\"{}\-Remote Logging\char`\"{} section.

\subsection*{Requirements}

The \hyperlink{interface_test_flight}{Test\-Flight} S\-D\-K requires i\-O\-S 4.\-3 or above, the Apple L\-L\-V\-M compiler, and the libz library to run.

The Ad\-Support.\-framework is required for i\-O\-S 6.\-0+ in order to uniquely identify users so we can estimate the number of users your app has (using {\ttfamily A\-S\-Identifier\-Manager}). You may weak link the framework in you app. If your app does not link with the Ad\-Support.\-framework, the \hyperlink{interface_test_flight}{Test\-Flight} S\-D\-K will automatically load it for apps running on i\-O\-S 6.\-0+.

\subsection*{Integration}


\begin{DoxyEnumerate}
\item Add the files to your project\-: File -\/$>$ Add Files to \char`\"{} \char`\"{}
\begin{DoxyEnumerate}
\item Find and select the folder that contains the S\-D\-K
\item Make sure that \char`\"{}\-Copy items into destination folder (if needed)\char`\"{} is checked
\item Set Folders to \char`\"{}\-Create groups for any added folders\char`\"{}
\item Select all targets that you want to add the S\-D\-K to
\end{DoxyEnumerate}
\item Verify that lib\-Test\-Flight.\-a has been added to the Link Binary With Libraries Build Phase for the targets you want to use the S\-D\-K with
\begin{DoxyEnumerate}
\item Select your Project in the Project Navigator
\item Select the target you want to enable the S\-D\-K for
\item Select the Build Phases tab
\item Open the Link Binary With Libraries Phase
\item If lib\-Test\-Flight.\-a is not listed, drag and drop the library from your Project Navigator to the Link Binary With Libraries area
\item Repeat Steps 2 -\/ 5 until all targets you want to use the S\-D\-K with have the S\-D\-K linked
\end{DoxyEnumerate}
\item Add libz to your Link Binary With Libraries Build Phase
\begin{DoxyEnumerate}
\item Select your Project in the Project Navigator
\item Select the target you want to enable the S\-D\-K for
\item Select the Build Phases tab
\item Open the Link Binary With Libraries Phase
\item Click the + to add a new library
\item Find libz.\-dylib in the list and add it
\item Repeat Steps 2 -\/ 6 until all targets you want to use the S\-D\-K with have libz.\-dylib
\end{DoxyEnumerate}
\item Get your App Token
\begin{DoxyEnumerate}
\item If this is a new application, and you have not uploaded it to \hyperlink{interface_test_flight}{Test\-Flight} before, first register it here\-: \mbox{[}\href{https://testflightapp.com/dashboard/applications/create/}{\tt https\-://testflightapp.\-com/dashboard/applications/create/}\mbox{]}().

Otherwise, if you have previously uploaded your app to \hyperlink{interface_test_flight}{Test\-Flight}, go to your list of applications (\mbox{[}\href{http://testflightapp.com/dashboard/applications/}{\tt http\-://testflightapp.\-com/dashboard/applications/}\mbox{]}()) and click on the application you are using from the list.
\item Click on the \char`\"{}\-App Token\char`\"{} tab on the left. The App Token for that application will be there.
\end{DoxyEnumerate}
\item In your Application Delegate\-:
\begin{DoxyEnumerate}
\item Import \hyperlink{interface_test_flight}{Test\-Flight}\-: {\ttfamily \#import \char`\"{}\-Test\-Flight.\-h\char`\"{}}
\item Launch \hyperlink{interface_test_flight}{Test\-Flight} with your App Token

In your {\ttfamily -\/application\-:did\-Finish\-Launching\-With\-Options\-:} method, call {\ttfamily +\mbox{[}\hyperlink{interface_test_flight}{Test\-Flight} take\-Off\-:\mbox{]}} with your App Token. \begin{DoxyVerb}-(BOOL)application:(UIApplication *)application 
    didFinishLaunchingWithOptions:(NSDictionary *)launchOptions {
  // start of your application:didFinishLaunchingWithOptions 

  [TestFlight takeOff:@"Insert your Application Token here"];

  // The rest of your application:didFinishLaunchingWithOptions method
  // ...
}
\end{DoxyVerb}

\item To report crashes to you we install our own uncaught exception handler. If you are not currently using an exception handler of your own then all you need to do is go to the next step. If you currently use an Exception Handler, or you use another framework that does please go to the section on advanced exception handling.
\end{DoxyEnumerate}
\end{DoxyEnumerate}

\subsection*{Setting the U\-D\-I\-D}

For {\bfseries B\-E\-T\-A} apps only\-: In order for \char`\"{}\-In App Updates\char`\"{} to work and for user data not to be anonymized, you may provide the device's unique identifier. To send the device identifier call the following method {\bfseries before} your call to {\ttfamily +\mbox{[}\hyperlink{interface_test_flight}{Test\-Flight} take\-Off\-:\mbox{]}} like so\-: \begin{DoxyVerb}[TestFlight setDeviceIdentifier:[[UIDevice currentDevice] uniqueIdentifier]];
[TestFlight takeOff:@"Insert your Application Token here"];
\end{DoxyVerb}


Note\-: {\ttfamily \mbox{[}\mbox{[}U\-I\-Device current\-Device\mbox{]} unique\-Identifier\mbox{]}} is deprecated, which means it may be removed from i\-O\-S in the future and that it should not be used in production apps. We recommend using it {\bfseries only} in beta apps. If using it makes you feel uncomfortable, you are not required to include it.

{\bfseries Note on i\-O\-S 7 and Xcode 5}\-: In i\-O\-S 7, {\ttfamily unique\-Identifier} no longer returns the device's U\-D\-I\-D, so i\-O\-S 7 users will show up anonymously on \hyperlink{interface_test_flight}{Test\-Flight}. Also, when building with A\-R\-C, Xcode 5 will not allow you to call {\ttfamily unique\-Identifier} because it has been removed in i\-O\-S 7 from {\ttfamily U\-I\-Device}'s header. We are working on a workaround for this issue.

{\bfseries D\-O N\-O\-T U\-S\-E T\-H\-I\-S I\-N P\-R\-O\-D\-U\-C\-T\-I\-O\-N A\-P\-P\-S}. When it is time to submit to the App Store comment this line out. Apple will probably reject your app if you leave this line in.

\subsection*{Uploading your build}

After you have integrated the S\-D\-K into your application you need to upload your build to \hyperlink{interface_test_flight}{Test\-Flight}. You can upload your build on our \href{https://testflightapp.com/dashboard/builds/add/}{\tt website}, using our \href{https://testflightapp.com/desktop/}{\tt desktop app}, or by using our \href{https://testflightapp.com/api/doc/}{\tt upload A\-P\-I}.

\subsection*{Basic Features}

\subsubsection*{Session Information}

View anonymous information about how often users use your app, how long they use it for, and when they use it. You can see what type of device the user is using, which O\-S, which language, etc.

Sessions automatically start at app launch, app did become active, and app will enter foreground and end at app will resign active, app did enter background, or app will terminate. Sessions that start shortly after an end continue the session instead of starting a new one.

For {\bfseries beta} users, you can see who the users are if you are {\bfseries setting the U\-D\-I\-D}, they have a \hyperlink{interface_test_flight}{Test\-Flight} account, and their device is registered to \hyperlink{interface_test_flight}{Test\-Flight}. (See Setting the U\-D\-I\-D for more information).

\subsubsection*{Crash Reports}

The \hyperlink{interface_test_flight}{Test\-Flight} S\-D\-K automatically reports all crashes (beta and prod) to \hyperlink{interface_test_flight}{Test\-Flight}'s website where you can view them. Crash reports are sent {\bfseries at} crash time. \hyperlink{interface_test_flight}{Test\-Flight} will also automatically symbolicate all crashes (if you have uploaded your d\-S\-Y\-M). For {\bfseries beta} apps, on the site, you can see which checkpoints the user passed before the crash and see remote logs that were sent before the crash. For {\bfseries prod} apps, you can see remote logs that were sent before the crash.

\subsubsection*{Beta In App Updates}

If a user is using a {\bfseries beta} version of your app, you are {\bfseries setting the U\-D\-I\-D}, a new beta version is available, and that user has permission to install it; an in app popup will ask them if they would like to install the update. If they tap \char`\"{}\-Install\char`\"{}, the new version is installed from inside the app.

N\-B\-: For this to work, you must increment your build version before uploading. Otherwise the new and old builds will have the same version number and we won't know if the user needs to update or is already using the new version.

To turn this off set this option before calling {\ttfamily take\-Off\-:} \begin{DoxyVerb}[TestFlight setOptions: TFOptionDisableInAppUpdates : @YES }];
\end{DoxyVerb}


\subsection*{Additional Features}

\subsubsection*{Checkpoints}

When a tester does something you care about in your app, you can pass a checkpoint. For example completing a level, adding a todo item, etc. The checkpoint progress is used to provide insight into how your testers are testing your apps. The passed checkpoints are also attached to crashes, which can help when creating steps to replicate. Checkpoints are visible for all beta and prod builds. \begin{DoxyVerb}[TestFlight passCheckpoint:@"CHECKPOINT_NAME"];
\end{DoxyVerb}


Use {\ttfamily pass\-Checkpoint\-:} to track when a user performs certain tasks in your application. This can be useful for making sure testers are hitting all parts of your application, as well as tracking which testers are being thorough.

Checkpoints are meant to tell you if a user visited a place in your app or completed a task. They should not be used for debugging purposes. Instead, use Remote Logging for debugging information (more information below).

\subsubsection*{Custom Environment Information}

In {\bfseries beta} builds, if you want to see some extra information about your user, you can add some custom environment information. You must add this information before the session starts (a session starts at {\ttfamily take\-Off\-:}) to see it on \hyperlink{interface_test_flight}{Test\-Flight}'s website. N\-B\-: You can only see this information for {\bfseries beta} users. \begin{DoxyVerb}[TestFlight addCustomEnvironmentInformation:@"info" forKey:@"key"];
\end{DoxyVerb}


You may call this method as many times as you would like to add more information.

\subsubsection*{User Feedback}

In {\bfseries beta} builds, if you collect feedback from your users, you may pass it back to \hyperlink{interface_test_flight}{Test\-Flight} which will associate it with the user's current session. \begin{DoxyVerb}[TestFlight submitFeedback:feedback];
\end{DoxyVerb}


Once users have submitted feedback from inside of the application you can view it in the feedback area of your build page.

\subsubsection*{Remote Logging}

Remote Logging allows you to see the logs your app prints out remotely, on \hyperlink{interface_test_flight}{Test\-Flight}'s website. You can see logs for {\bfseries beta sessions} and {\bfseries prod sessions with crashes}. N\-B\-: you cannot see the logs for all prod sessions.

To use it, simply replace all of your {\ttfamily N\-S\-Log} calls with {\ttfamily T\-F\-Log} calls. An easy way to do this without rewriting all your {\ttfamily N\-S\-Log} calls is to add the following macro to your {\ttfamily .pch} file. \begin{DoxyVerb}#import "TestFlight.h"
#define NSLog TFLog
\end{DoxyVerb}


Not only will {\ttfamily T\-F\-Log} log remotely to \hyperlink{interface_test_flight}{Test\-Flight}, it will also log to the console (viewable in a device's logs) and S\-T\-D\-E\-R\-R (shown while debugging) just like N\-S\-Log does, providing a complete replacement.

For even better information in your remote logs, such as file name and line number, you can use this macro instead\-: \begin{DoxyVerb}#define NSLog(__FORMAT__, ...) TFLog((@"%s [Line %d] " __FORMAT__), __PRETTY_FUNCTION__, __LINE__, ##__VA_ARGS__)
\end{DoxyVerb}


Which will produce output that looks like \begin{DoxyVerb}-[MyAppDelegate application:didFinishLaunchingWithOptions:] [Line 45] Launched!
\end{DoxyVerb}


{\bfseries Custom Logging}

If you have your own custom logging, call {\ttfamily T\-F\-Log} from your custom logging function. If you do not need {\ttfamily T\-F\-Log} to log to the console or S\-T\-D\-E\-R\-R because you handle those yourself, you can turn them off with these calls\-: \begin{DoxyVerb}[TestFlight setOptions: TFOptionLogToConsole : @NO }];
[TestFlight setOptions: TFOptionLogToSTDERR : @NO }];
\end{DoxyVerb}


\subsection*{Advanced Notes}

\subsubsection*{Checkpoint A\-P\-I}

When passing a checkpoint, \hyperlink{interface_test_flight}{Test\-Flight} logs the checkpoint synchronously (See Remote Logging for more information). If your app has very high performance needs, you can turn the logging off with the {\ttfamily T\-F\-Option\-Log\-On\-Checkpoint} option.

\subsubsection*{Remote Logging}

All logging is done synchronously. Every time the S\-D\-K logs, it must write data to a file. This is to ensure log integrity at crash time. Without this, we could not trust logs at crash time. If you have a high performance app, please email \href{mailto:support@testflightapp.com}{\tt support@testflightapp.\-com} for more options.

\subsubsection*{Advanced Session Control}

Continuing sessions\-: You can adjust the amount of time a user can leave the app for and still continue the same session when they come back by changing the {\ttfamily T\-F\-Option\-Session\-Keep\-Alive\-Timeout} option. Change it to 0 to turn the feature off.

Manual Session Control\-: If your app is a music player that continues to play music in the background, a navigation app that continues to function in the background, or any app where a user is considered to be \char`\"{}using\char`\"{} the app even while the app is not active you should use Manual Session Control. Please only use manual session control if you know exactly what you are doing. There are many pitfalls which can result in bad session duration and counts. See {\ttfamily Test\-Flight+\-Manual\-Sessions.h} for more information and instructions.

\subsubsection*{Advanced Exception/\-Signal Handling}

An uncaught exception means that your application is in an unknown state and there is not much that you can do but try and exit gracefully. Our S\-D\-K does its best to get the data we collect in this situation to you while it is crashing, but it is designed in such a way that the important act of saving the data occurs in as safe way a way as possible before trying to send anything. If you do use uncaught exception or signal handlers, install your handlers before calling {\ttfamily take\-Off\-:}. Our S\-D\-K will then call your handler while ours is running. For example\-: \begin{DoxyVerb}  /*
   My Apps Custom uncaught exception catcher, we do special stuff here, and TestFlight takes care of the rest
  */
  void HandleExceptions(NSException *exception) {
    NSLog(@"This is where we save the application data during a exception");
    // Save application data on crash
  }
  /*
   My Apps Custom signal catcher, we do special stuff here, and TestFlight takes care of the rest
  */
  void SignalHandler(int sig) {
    NSLog(@"This is where we save the application data during a signal");
    // Save application data on crash
  }

  -(BOOL)application:(UIApplication *)application 
  didFinishLaunchingWithOptions:(NSDictionary *)launchOptions {    
    // installs HandleExceptions as the Uncaught Exception Handler
    NSSetUncaughtExceptionHandler(&HandleExceptions);
    // create the signal action structure 
    struct sigaction newSignalAction;
    // initialize the signal action structure
    memset(&newSignalAction, 0, sizeof(newSignalAction));
    // set SignalHandler as the handler in the signal action structure
    newSignalAction.sa_handler = &SignalHandler;
    // set SignalHandler as the handlers for SIGABRT, SIGILL and SIGBUS
    sigaction(SIGABRT, &newSignalAction, NULL);
    sigaction(SIGILL, &newSignalAction, NULL);
    sigaction(SIGBUS, &newSignalAction, NULL);
    // Call takeOff after install your own unhandled exception and signal handlers
    [TestFlight takeOff:@"Insert your Application Token here"];
    // continue with your application initialization
  }
\end{DoxyVerb}


You do not need to add the above code if your application does not use exception handling already. 