The \hyperlink{interface_test_flight}{Test\-Flight} S\-D\-K allows you to track how beta testers are testing your application. Out of the box we track simple usage information, such as which tester is using your application, their device model/\-O\-S, how long they used the application, logs of their test session, and automatic recording of any crashes they encounter.

To get the most out of the S\-D\-K we have provided the Checkpoint A\-P\-I.

The Checkpoint A\-P\-I is used to help you track exactly how your testers are using your application. Curious about which users passed level 5 in your game, or posted their high score to Twitter, or found that obscure feature? With a single line of code you can finally gather all this information. Wondering how many times your app has crashed? Wondering who your power testers are? We've got you covered. See more information on the Checkpoint A\-P\-I in section 4.

Alongside the Checkpoint A\-P\-I is the Questions interface. The Questions interface is managed on a per build basis on the \hyperlink{interface_test_flight}{Test\-Flight} website. Find out more about the Questions Interface in section 6.

For more detailed debugging we have a remote logging solution. Find out more about our logging system with T\-F\-Log in the Remote Logging section.

\subsection*{Considerations}

Information gathered by the S\-D\-K is sent to the website in real time. When an application is put into the background (i\-O\-S 4.\-x) or terminated (i\-O\-S 3.\-x) we try to send the finalizing information for the session during the time allowed for finalizing the application. Should all of the data not get sent the remaining data will be sent the next time the application is launched. As such, to get the most out of the S\-D\-K we recommend your application support i\-O\-S 4.\-0 and higher.

This S\-D\-K can be run from both the i\-Phone Simulator and Device and has been tested using Xcode 4.\-0.

\subsection*{Integration}


\begin{DoxyEnumerate}
\item Add the files to your project\-: File -\/$>$ Add Files to \char`\"{} \char`\"{}
\begin{DoxyEnumerate}
\item Find and select the folder that contains the S\-D\-K
\item Make sure that \char`\"{}\-Copy items into destination folder (if needed)\char`\"{} is checked
\item Set Folders to \char`\"{}\-Create groups for any added folders\char`\"{}
\item Select all targets that you want to add the S\-D\-K to
\end{DoxyEnumerate}
\item Verify that lib\-Test\-Flight.\-a has been added to the Link Binary With Libraries Build Phase for the targets you want to use the S\-D\-K with
\begin{DoxyEnumerate}
\item Select your Project in the Project Navigator
\item Select the target you want to enable the S\-D\-K for
\item Select the Build Phases tab
\item Open the Link Binary With Libraries Phase
\item If lib\-Test\-Flight.\-a is not listed, drag and drop the library from your Project Navigator to the Link Binary With Libraries area
\item Repeat Steps 2 -\/ 5 until all targets you want to use the S\-D\-K with have the S\-D\-K linked
\end{DoxyEnumerate}
\item Add libz to your Link Binary With Libraries Build Phase
\begin{DoxyEnumerate}
\item Select your Project in the Project Navigator
\item Select the target you want to enable the S\-D\-K for
\item Select the Build Phases tab
\item Open the Link Binary With Libraries Phase
\item Click the + to add a new library
\item Find libz.\-dylib in the list and add it
\item Repeat Steps 2 -\/ 6 until all targets you want to use the S\-D\-K with have libz.\-dylib
\end{DoxyEnumerate}
\end{DoxyEnumerate}


\begin{DoxyEnumerate}
\item In your Application Delegate\-:
\begin{DoxyEnumerate}
\item Import \hyperlink{interface_test_flight}{Test\-Flight}\-: {\ttfamily \#import \char`\"{}\-Test\-Flight.\-h\char`\"{}}

$\ast$$\ast$$\ast$\-N\-O\-T\-E\-:$\ast$$\ast$$\ast$ Rather than importing {\ttfamily \hyperlink{_test_flight_8h}{Test\-Flight.\-h}} in every file you may add the above line into you pre-\/compiled header ({\ttfamily $<$projectname$>$\-\_\-\-Prefix.\-pch}) file inside of the \begin{DoxyVerb}#ifdef __OBJC__ 
\end{DoxyVerb}


section. This will give you access to the S\-D\-K across all files.
\end{DoxyEnumerate}


\begin{DoxyEnumerate}
\item Get your Team Token which you can find at \href{http://testflightapp.com/dashboard/team/}{\tt http\-://testflightapp.\-com/dashboard/team/} select the team you are using from the team selection drop down list on the top of the page and then select Team Info.
\end{DoxyEnumerate}


\begin{DoxyEnumerate}
\item Launch \hyperlink{interface_test_flight}{Test\-Flight} with your Team Token \begin{DoxyVerb} -(BOOL)application:(UIApplication *)application 
     didFinishLaunchingWithOptions:(NSDictionary *)launchOptions {
 // start of your application:didFinishLaunchingWithOptions 
 // ...
 [TestFlight takeOff:@"Insert your Team Token here"];
 // The rest of your application:didFinishLaunchingWithOptions method
 // ...
 }
\end{DoxyVerb}

\end{DoxyEnumerate}


\begin{DoxyEnumerate}
\item To report crashes to you we install our own uncaught exception handler. If you are not currently using an exception handler of your own then all you need to do is go to the next step. If you currently use an Exception Handler, or you use another framework that does please go to the section on advanced exception handling.
\end{DoxyEnumerate}
\end{DoxyEnumerate}


\begin{DoxyEnumerate}
\item To enable the best crash reporting possible we recommend setting the following project build settings in Xcode to N\-O for all targets that you want to have live crash reporting for. You can find build settings by opening the Project Navigator (default command+1 or command+shift+j) then clicking on the project you are configuring (usually the first selection in the list). From there you can choose to either change the global project settings or settings on an individual project basis. All settings below are in the Deployment Section.


\begin{DoxyEnumerate}
\item Deployment Post Processing
\item Strip Debug Symbols During Copy
\item Strip Linked Product
\end{DoxyEnumerate}
\end{DoxyEnumerate}

\subsection*{Beta Testing and Release Differentiation}

In order to provide more information about your testers while beta testing you will need to provide the device's unique identifier. This identifier is not something that the S\-D\-K will collect from the device and we do not recommend using this in production. To send the device identifier to us put the following code before your call to take\-Off. \begin{DoxyVerb}#define TESTING 1
#ifdef TESTING
    [TestFlight setDeviceIdentifier:[[UIDevice currentDevice] uniqueIdentifier]];
#endif
\end{DoxyVerb}


This will allow you to have the best possible information during testing, but disable getting and sending of the device unique identifier when you release your application. When it is time to release simply comment out \#define T\-E\-S\-T\-I\-N\-G 1. If you decide to not include the device's unique identifier during your testing phase \hyperlink{interface_test_flight}{Test\-Flight} will still collect all of the information that you send but it may be anonymized.

\subsection*{Checkpoint A\-P\-I}

When a tester does something you care about in your app you can pass a checkpoint. For example completing a level, adding a todo item, etc. The checkpoint progress is used to provide insight into how your testers are testing your apps. The passed checkpoints are also attached to crashes, which can help when creating steps to replicate.

{\ttfamily \mbox{[}\hyperlink{interface_test_flight}{Test\-Flight} pass\-Checkpoint\-:@\char`\"{}\-C\-H\-E\-C\-K\-P\-O\-I\-N\-T\-\_\-\-N\-A\-M\-E\char`\"{}\mbox{]};} Use {\ttfamily pass\-Checkpoint\-:} to track when a user performs certain tasks in your application. This can be useful for making sure testers are hitting all parts of your application, as well as tracking which testers are being thorough.

\subsection*{Feedback A\-P\-I}

To launch unguided feedback call the open\-Feedback\-View method. We recommend that you call this from a G\-U\-I element. \begin{DoxyVerb}-(IBAction)launchFeedback {
    [TestFlight openFeedbackView];
}
\end{DoxyVerb}


If you want to create your own feedback form you can use the submit\-Custom\-Feedback method to submit the feedback that the user has entered. \begin{DoxyVerb}-(IBAction)submitFeedbackPressed:(id)sender {
    NSString *feedback = [self getUserFeedback];
    [TestFlight submitFeedback:feedback];
}
\end{DoxyVerb}


The above sample assumes that \mbox{[}self get\-User\-Feedback\mbox{]} is implemented such that it obtains the users feedback from the G\-U\-I element you have created and that submit\-Feedback\-Pressed is the action for your submit button.

Once users have submitted feedback from inside of the application you can view it in the feedback area of your build page.

\subsection*{Upload your build}

After you have integrated the S\-D\-K into your application you need to upload your build to \hyperlink{interface_test_flight}{Test\-Flight}. You can upload from your dashboard or or using the Upload A\-P\-I, full documentation at \href{https://testflightapp.com/api/doc/}{\tt https\-://testflightapp.\-com/api/doc/}

\subsection*{Questions Interface}

In order to ask a question, you'll need to associate it with a checkpoint. Make sure your checkpoints are initialized by running your app and hitting them all yourself before you start adding questions.

There are three question types available\-: Yes/\-No, Multiple Choice, and Long Answer.

To create questions, visit your builds Questions page and click on 'Add Question'. If you choose Multiple Choice, you'll need to enter a list of possible answers for your testers to choose from — otherwise, you'll only need to enter your question's, well, question. If your build has no questions, you can also choose to migrate questions from another build (because seriously — who wants to do all that typing again)?

After restarting your application on an approved device, when you pass the checkpoint associated with your questions a \hyperlink{interface_test_flight}{Test\-Flight} modal question form will appear on the screen asking the beta tester to answer your question.

After you upload a new build to \hyperlink{interface_test_flight}{Test\-Flight} you will need to associate questions once again. However if your checkpoints and questions have remained the same you can choose \char`\"{}copy questions from an older build\char`\"{} and choose which build to copy the questions from.

\subsection*{View the results}

As testers install your build and start to test it you will see their session data on the web on the build report page for the build you've uploaded.

\subsection*{Advanced Exception Handling}

An uncaught exception means that your application is in an unknown state and there is not much that you can do but try and exit gracefully. Our S\-D\-K does its best to get the data we collect in this situation to you while it is crashing, but it is designed in such a way that the important act of saving the data occurs in as safe way a way as possible before trying to send anything. If you do use uncaught exception or signal handlers install your handlers before calling {\ttfamily take\-Off}. Our S\-D\-K will then call your handler while ours is running. For example\-: \begin{DoxyVerb}  /*
   My Apps Custom uncaught exception catcher, we do special stuff here, and TestFlight takes care of the rest
  */
  void HandleExceptions(NSException *exception) {
    NSLog(@"This is where we save the application data during a exception");
    // Save application data on crash
  }
  /*
   My Apps Custom signal catcher, we do special stuff here, and TestFlight takes care of the rest
  */
  void SignalHandler(int sig) {
    NSLog(@"This is where we save the application data during a signal");
    // Save application data on crash
  }

  -(BOOL)application:(UIApplication *)application 
  didFinishLaunchingWithOptions:(NSDictionary *)launchOptions {    
    // installs HandleExceptions as the Uncaught Exception Handler
    NSSetUncaughtExceptionHandler(&HandleExceptions);
    // create the signal action structure 
    struct sigaction newSignalAction;
    // initialize the signal action structure
    memset(&newSignalAction, 0, sizeof(newSignalAction));
    // set SignalHandler as the handler in the signal action structure
    newSignalAction.sa_handler = &SignalHandler;
    // set SignalHandler as the handlers for SIGABRT, SIGILL and SIGBUS
    sigaction(SIGABRT, &newSignalAction, NULL);
    sigaction(SIGILL, &newSignalAction, NULL);
    sigaction(SIGBUS, &newSignalAction, NULL);
    // Call takeOff after install your own unhandled exception and signal handlers
    [TestFlight takeOff:@"Insert your Team Token here"];
    // continue with your application initialization
  }
\end{DoxyVerb}


You do not need to add the above code if your application does not use exception handling already.

\subsection*{Remote Logging}

To perform remote logging you can use the T\-F\-Log method which logs in a few different methods described below. In order to make the transition from N\-S\-Log to T\-F\-Log easy we have used the same method signature for T\-F\-Log as N\-S\-Log. You can easily switch over to T\-F\-Log by adding the following macro to your header \begin{DoxyVerb}#define NSLog TFLog
\end{DoxyVerb}


That will do a switch from N\-S\-Log to T\-F\-Log, if you want more information, such as file name and line number you can use a macro like \begin{DoxyVerb}#define NSLog(__FORMAT__, ...) TFLog((@"%s [Line %d] " __FORMAT__), __PRETTY_FUNCTION__, __LINE__, ##__VA_ARGS__)
\end{DoxyVerb}


Which will produce output that looks like \begin{DoxyVerb}-[HTFCheckpointsController showYesNoQuestion:] [Line 45] Pressed YES/NO
\end{DoxyVerb}


We have implemented three different loggers. \begin{DoxyVerb}1. TestFlight logger
2. Apple System Log logger
3. STDERR logger
\end{DoxyVerb}


Each of the loggers log asynchronously and all T\-F\-Log calls are non blocking. The \hyperlink{interface_test_flight}{Test\-Flight} logger writes its data to a file which is then sent to our servers on Session End events. The Apple System Logger sends its messages to the Apple System Log and are viewable using the Organizer in Xcode when the device is attached to your computer. The A\-S\-L logger can be disabled by turning it off in your \hyperlink{interface_test_flight}{Test\-Flight} options \begin{DoxyVerb}[TestFlight setOptions:[NSDictionary dictionaryWithObject:[NSNumber numberWithBool:NO] forKey:@"logToConsole"]];
\end{DoxyVerb}


The default option is Y\-E\-S.

The S\-T\-D\-E\-R\-R logger sends log messages to S\-T\-D\-E\-R\-R so that you can see your log statements while debugging. The S\-T\-D\-E\-R\-R logger is only active when a debugger is attached to your application. If you do not wish to use the S\-T\-D\-E\-R\-R logger you can disable it by turning it off in your \hyperlink{interface_test_flight}{Test\-Flight} options \begin{DoxyVerb}[TestFlight setOptions:[NSDictionary dictionaryWithObject:[NSNumber numberWithBool:NO] forKey:@"logToSTDERR"]];
\end{DoxyVerb}


The default option is Y\-E\-S.

\subsection*{Advanced Remote Logging}

For most users we expect using T\-F\-Log to provide all of the logging functionality that they need. For the occasion where you need to provide a wrapper around T\-F\-Log we provide \begin{DoxyVerb}void TFLogv(NSString *format, va_list arg_list);
\end{DoxyVerb}


Using T\-F\-Logv you can have your method that accepts a variable number of arguments that then passes that format and argument list to T\-F\-Log.

\subsection*{i\-O\-S3}

We now require that anyone who is writing an application that supports i\-O\-S3 add the System.\-framework as an optional link. In order to provide a better shutdown experience we send any large log files to our servers in the background. To add System.\-framework as an optional link\-:


\begin{DoxyEnumerate}
\item Select your Project in the Project Navigator
\item Select the target you want to enable the S\-D\-K for
\item Select the Build Phases tab
\item Open the Link Binary With Libraries Phase
\item Click the + to add a new library
\item Find lib\-System.\-dylib in the list and add it
\item To the right of lib\-System.\-dylib in the Link Binary With Libraries pane change \char`\"{}\-Required\char`\"{} to \char`\"{}\-Optional\char`\"{} 
\end{DoxyEnumerate}